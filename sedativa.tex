\begin{frame}{
    \textbf{Midazolam}
    \textit{(Dormicum)}
}
    \begin{tabular}{c c}
        \begin{beamercolorbox}[wd=\boxwidth\textwidth,ht=\boxheight\textheight,sep=1em]{indikation}
            \begin{itemize}
                \item Krampfanfälle (Status epilepticus)
                \item Narkoseeinleitung
            \end{itemize}
        \end{beamercolorbox} & 
        \begin{beamercolorbox}[wd=\boxwidth\textwidth,ht=\boxheight\textheight,sep=1em]{kontraindikation}
            \begin{itemize}
                \item // todo Kontraindikation 
            \end{itemize}
        \end{beamercolorbox} \\
        \begin{beamercolorbox}[wd=\boxwidth\textwidth,ht=\boxheight\textheight,sep=1em]{wirkung}
            \scriptsize
            \begin{itemize}
                \item Benzodiazepin mit sedierender, anxiolytischer, antikonvulsier und muskelrelaxierender Wirkung
                \item Wirkungseintritt nach ca. 2-3min, Wirkungsdauer: 45min, HWZ ca. 5 Stunden, gut steuerbar
            \end{itemize}
        \end{beamercolorbox} & 
        \begin{beamercolorbox}[wd=\boxwidth\textwidth,ht=\boxheight\textheight,sep=1em]{nebenwirkung}
            \scriptsize
            \begin{itemize}
                \item Blutdruckabfall (gering)
                \item Atemdepression
                \item ZNS-Störungen und paradoxe Wirkung
                \item anterograde Amnesie
                \item wirkt ev. stärker oder auch paradox bei älteren Patienten (über 60a)
            \end{itemize}
        \end{beamercolorbox} \\
    \end{tabular}
\end{frame}

\begin{frame}{
    \textbf{Midazolam}
    \textit{(Dormicum)}
}
    \begin{tabular}{c c}
        \begin{beamercolorbox}[wd=\boxwidth\textwidth,ht=\boxheight\textheight,sep=1em]{dosierung}
            \scriptsize
            \begin{itemize}
                \item Prämediaktion Erwachsene: 0,05-0,1\mgkgkg
                \item Status epilepticus: 0,2\mgkgkg langsam i.v. bzw. i.m
                \item ggf. ist eine Dosisreduktion (Kinder und ältere Patienten) oder Steigerung der Dosis erforderlich
                \item orale, rektale oder nasale Gabe möglich
            \end{itemize}
        \end{beamercolorbox} & 
        \begin{beamercolorbox}[wd=\boxwidth\textwidth,ht=\boxheight\textheight,sep=1em]{zusammensetzung}
            \begin{itemize}
                \item 1 Amp. mit 3ml enthält 15mg Midazolam (3\mgml)
                \item 1 Amp. mit 10ml enthält 50mg Midazolam (5\mgml)
            \end{itemize}
        \end{beamercolorbox} \\
        \multicolumn{2}{}{}
        \begin{beamercolorbox}[wd=\textwidth,ht=\boxheight\textheight,sep=1em]{bemerkung}
            \begin{itemize}
                \item Wechselwirkungen: Wirkungsverstärkung mit zentraldämpfenden Medikamenten (Alkohol!)
            \end{itemize}
        \end{beamercolorbox} \\
    \end{tabular}
\end{frame}

\begin{frame}{
    \textbf{Diazepam}
    \textit{(Gewacalm, Psychopax)}
}
    \begin{tabular}{c c}
        \begin{beamercolorbox}[wd=\boxwidth\textwidth,ht=\boxheight\textheight,sep=1em]{indikation}
            \begin{itemize}
                \item Erregungszustände, Sedierung
                \item Krampfanfälle
            \end{itemize}
        \end{beamercolorbox} & 
        \begin{beamercolorbox}[wd=\boxwidth\textwidth,ht=\boxheight\textheight,sep=1em]{kontraindikation}
            \begin{itemize}
                \item // todo Kontraindikation 
            \end{itemize}
        \end{beamercolorbox} \\
        \begin{beamercolorbox}[wd=\boxwidth\textwidth,ht=\boxheight\textheight,sep=1em]{wirkung}
            \scriptsize
            \begin{itemize}
                \item Benzodiazepin mit sedierender, anxiolytischer, antikonvulsier und muskelrelaxierender Wirkung
                \item Wirkeintritt sofort
                \item Wirkdauer: 15min bis zu Stunden
                \item HWZ: ca. 30h
                \item \textbf{CAVE: Kumulation!} Patienten die Diazepam i.v. erhalten, müssen überwacht werden!
            \end{itemize}
        \end{beamercolorbox} & 
        \begin{beamercolorbox}[wd=\boxwidth\textwidth,ht=\boxheight\textheight,sep=1em]{nebenwirkung}
            \scriptsize
            \begin{itemize}
                \item Blutdruckabfall (gering)
                \item Atemdepression
                \item ZNS-Störungen und paradoxe Wirkung
                \item Venenreizung, Thrombophlebitiden, Nekrose
            \end{itemize}
        \end{beamercolorbox} \\
    \end{tabular}
\end{frame}

\begin{frame}{
    \textbf{Diazepam}
    \textit{(Gewacalm, Psychopax)}
}
    \begin{tabular}{c c}
        \begin{beamercolorbox}[wd=\boxwidth\textwidth,ht=\boxheight\textheight,sep=1em]{dosierung}
            \scriptsize
            \begin{itemize}
                \item 1 Amp. (10mg) langsam i.v. bis zur gewünschten Wirkung ("Schließen der Augen")
                \item ggf. ist eine Dosisreduktion (Kinder und ältere Patienten) oder Steigerung der Dosis erforderlich
                \item 0,1-0,5\mgkgkg, Tageshöchstdosis: 100mg
            \end{itemize}
        \end{beamercolorbox} & 
        \begin{beamercolorbox}[wd=\boxwidth\textwidth,ht=\boxheight\textheight,sep=1em]{zusammensetzung}
            \begin{itemize}
                \item 1 Amp. mit 2ml enthält 10mg Diazepam (5\mgml)
            \end{itemize}
        \end{beamercolorbox} \\
        \multicolumn{2}{}{}
        \begin{beamercolorbox}[wd=\textwidth,ht=\boxheight\textheight,sep=1em]{bemerkung}
            \begin{itemize}
                \scriptsize
                \item {
                    Wechselwirkungen
                    \begin{itemize}
                        \scriptsize
                        \item zentral dämpfende Medikamente (Wirkungsverstärkung)
                        \item Muskelrelaxantien (Wirkungsverlängerung)
                    \end{itemize}
                }
                \item {
                    Inkompatibilitäten
                    \begin{itemize}
                        \scriptsize
                        \item grundsätzlich alleine injizieren, da mit vielen Arzneistoffen unverträglich
                    \end{itemize}
                }
            \end{itemize}
        \end{beamercolorbox} \\
    \end{tabular}
\end{frame}

\begin{frame}{
    \textbf{Diazepam}
    \textit{(Stesolid)}
}
    \begin{tabular}{c c}
        \begin{beamercolorbox}[wd=\boxwidth\textwidth,ht=\boxheight\textheight,sep=1em]{indikation}
            \begin{itemize}
                \item Krampfanfälle, v.a. Fieberkrampf
                \item Sedierung von Kindern
            \end{itemize}
        \end{beamercolorbox} & 
        \begin{beamercolorbox}[wd=\boxwidth\textwidth,ht=\boxheight\textheight,sep=1em]{kontraindikation}
            \begin{itemize}
                \item // todo Kontraindikation 
            \end{itemize}
        \end{beamercolorbox} \\
        \begin{beamercolorbox}[wd=\boxwidth\textwidth,ht=\boxheight\textheight,sep=1em]{wirkung}
            \begin{itemize}
                \item Benzodiazepin mit sedierender, anxiolytischer und antikonvulsier Wirkung
            \end{itemize}
        \end{beamercolorbox} & 
        \begin{beamercolorbox}[wd=\boxwidth\textwidth,ht=\boxheight\textheight,sep=1em]{nebenwirkung}
            \begin{itemize}
                \item Blutdruckabfall
                \item Atemdepression
            \end{itemize}
        \end{beamercolorbox} \\
    \end{tabular}
\end{frame}

\begin{frame}{
    \textbf{Diazepam}
    \textit{(Stesolid)}
}
    \begin{tabular}{c c}
        \begin{beamercolorbox}[wd=\boxwidth\textwidth,ht=\boxheight\textheight,sep=1em]{dosierung}
            \begin{itemize}
                \item $<$ 15kg Körpergewicht - 5mg Diazepam
                \item $>$ 15kg Körpergewicht - 10mg Diazepam
            \end{itemize}
        \end{beamercolorbox} & 
        \begin{beamercolorbox}[wd=\boxwidth\textwidth,ht=\boxheight\textheight,sep=1em]{zusammensetzung}
            \begin{itemize}
                \item 1 Rektiole enthälte 5mg oder 10mg Diazepam
            \end{itemize}
        \end{beamercolorbox} \\
        \multicolumn{2}{}{}
        \begin{beamercolorbox}[wd=\textwidth,ht=\boxheight\textheight,sep=1em]{bemerkung}
            \begin{itemize}
                \item Rektaltube bis zur Gewichtsmarkierung einführen
                \item zusammendrück und zusammengedrückt wieder entfernen, der in der Tube verbleibende Rest ist miteinkalkuliert
                \item im Notfall kann z.B.: ein wenig Xylacoingel al Gleitmittel verwendet werden
            \end{itemize}
        \end{beamercolorbox} \\
    \end{tabular}
\end{frame}

\begin{frame}{
    \textbf{Lorazepam}
    \textit{(Temesta)}
}
    \begin{tabular}{c c}
        \begin{beamercolorbox}[wd=\boxwidth\textwidth,ht=\boxheight\textheight,sep=1em]{indikation}
            \begin{itemize}
                \item Angstzustände (auch Alkohol-Entzug und Delir)
                \item Krampfanfälle (Status epilepticus)
                \item Prämediaktion
            \end{itemize}
        \end{beamercolorbox} & 
        \begin{beamercolorbox}[wd=\boxwidth\textwidth,ht=\boxheight\textheight,sep=1em]{kontraindikation}
            \begin{itemize}
                \item // todo Kontraindikation 
            \end{itemize}
        \end{beamercolorbox} \\
        \begin{beamercolorbox}[wd=\boxwidth\textwidth,ht=\boxheight\textheight,sep=1em]{wirkung}
            \scriptsize
            \begin{itemize}
                \item Benzodiazepin mit sedierender, anxiolytischer Wirkung bereits in geringer Dosierung
                \item bei höherer Dosierung auch relaxierend und antikonvulsiv
                \item Wirkungsdauer: HWZ ca. 14 Stunden
            \end{itemize}
        \end{beamercolorbox} & 
        \begin{beamercolorbox}[wd=\boxwidth\textwidth,ht=\boxheight\textheight,sep=1em]{nebenwirkung}
            \tiny
            \begin{itemize}
                \item Müdigkeit, Schläfrigkeit
                \item hyper-/hypotone Kreislaufreaktion
                \item Sehstörungen, trockener Mund
                \item Atemdepression
                \item ZSN-Störungen und paradoxe Wirkung
                \item anterograde Amnesie
            \end{itemize}
        \end{beamercolorbox} \\
    \end{tabular}
\end{frame}

\begin{frame}{
    \textbf{Lorazepam}
    \textit{(Temesta)}
}
    \begin{tabular}{c c}
        \begin{beamercolorbox}[wd=\boxwidth\textwidth,ht=\boxheight\textheight,sep=1em]{dosierung}
            \tiny
            \begin{itemize}
                \item Temesta-Ampullen z.B.: im Verhältnis 1:1 verdünnen
                \item Sedierung: bis zu 4mg langsam titrieren
                \item zur Krampfunterbrechung beim Status epilepticus gibt man 4mg langsam i.v. (über 2min)
                \item bei Anhalten bzw. Wiederkehr der Anfälle erneute Gabe von 4mg nach ca. 10-15min
            \end{itemize}
        \end{beamercolorbox} & 
        \begin{beamercolorbox}[wd=\boxwidth\textwidth,ht=\boxheight\textheight,sep=1em]{zusammensetzung}
            \begin{itemize}
                \item 1 Amp. mit 1ml enthält 4mg Lorazepam
            \end{itemize}
        \end{beamercolorbox} \\
        \multicolumn{2}{}{}
        \begin{beamercolorbox}[wd=\textwidth,ht=\boxheight\textheight,sep=1em]{bemerkung}
            \begin{itemize}
                \item zentraldämpfenden Medikamente, Alkohol, Analgetika (Wirkungsverstärkung)
            \end{itemize}
        \end{beamercolorbox} \\
    \end{tabular}
\end{frame}

\begin{frame}{
    \textbf{Haloperidol}
    \textit{(Haldol)}
}
    \begin{tabular}{c c}
        \begin{beamercolorbox}[wd=\boxwidth\textwidth,ht=\boxheight\textheight,sep=1em]{indikation}
            \scriptsize
            \begin{itemize}
                \item Unruhe-, Angst- und Erregungszustände
                \item Psychose, Wahn, Halluzinationen
                \item Hyperkinesien
                \item starkes Erbrechen
                \item Sedierung bei Suchmittelentzug
            \end{itemize}
        \end{beamercolorbox} & 
        \begin{beamercolorbox}[wd=\boxwidth\textwidth,ht=\boxheight\textheight,sep=1em]{kontraindikation}
            \scriptsize
            \begin{itemize}
                \item Epilepsie
                \item Morbus Parkinson
                \item Schädelhirntrauma
                \item Schwangerschaft
            \end{itemize}
        \end{beamercolorbox} \\
        \begin{beamercolorbox}[wd=\boxwidth\textwidth,ht=\boxheight\textheight,sep=1em]{wirkung}
            \scriptsize
            \begin{itemize}
                \item Neuroleptikum mit starker antipsychotischer, starker antiemetischer und gering sedierender Wirkung
                \item =hochpotentes Neuroleptikum
                \item Wirkeintritt nach ca. 10-15min i.v., nach ca. 20min i.m.
                \item Wirkdauer ca. 5-8 Stunden; HWZ: 12-36h
            \end{itemize}
        \end{beamercolorbox} & 
        \begin{beamercolorbox}[wd=\boxwidth\textwidth,ht=\boxheight\textheight,sep=1em]{nebenwirkung}
            \scriptsize
            \begin{itemize}
                \item Dyskinesien
                \item Mundtrockenheit, Sehstörungen
                \item Erhöhung der Krampfbereitschaft
                \item Blutdruckabfall
                \item Herzryhtmusstörungen, QT-Zeit-Verlängerungen
            \end{itemize}
        \end{beamercolorbox} \\
    \end{tabular}
\end{frame}

\begin{frame}{
    \textbf{Haloperidol}
    \textit{(Haldol)}
}
    \begin{tabular}{c c}
        \begin{beamercolorbox}[wd=\boxwidth\textwidth,ht=\boxheight\textheight,sep=1em]{dosierung}
            \begin{itemize}
                \item $\frac{1}{2}$-1 Ampulle langsam i.v. (2,5-5mg)
            \end{itemize}
        \end{beamercolorbox} & 
        \begin{beamercolorbox}[wd=\boxwidth\textwidth,ht=\boxheight\textheight,sep=1em]{zusammensetzung}
            \begin{itemize}
                \item 1 Amp. mit 1ml enthält 5mg Haloperidol
            \end{itemize}
        \end{beamercolorbox} \\
        \multicolumn{2}{}{}
        \begin{beamercolorbox}[wd=\textwidth,ht=\boxheight\textheight,sep=1em]{bemerkung}
            Wechselwirkungen
            \begin{itemize}
                \item Antihypertonika und zentraldämpfende Pharmaka (Wirkungsverstärkung)
                \item Adrenalin: paradoxe Hypotonie
            \end{itemize}
        \end{beamercolorbox} \\
    \end{tabular}
\end{frame}

\begin{frame}{
    \textbf{Dehydrobenzperidol}
    \textit{(PONVeridol)}
}
    \begin{tabular}{c c}
        \begin{beamercolorbox}[wd=\boxwidth\textwidth,ht=\boxheight\textheight,sep=1em]{indikation}
            \begin{itemize}
                \item Übelkeit
                \item Erbrechen
                \item PONV
            \end{itemize}
        \end{beamercolorbox} & 
        \begin{beamercolorbox}[wd=\boxwidth\textwidth,ht=\boxheight\textheight,sep=1em]{kontraindikation}
            \scriptsize
            \begin{itemize}
                \item akute Intoxikation mit zentral dämpfenden Substanzen (Alkohol, Sedativa) 
                \item Analgetikaintoxikationen
                \item Psychopharmaka-Intoxikationen
                \item relative Kontraindikation: Schwangerschaft
            \end{itemize}
        \end{beamercolorbox} \\
        \begin{beamercolorbox}[wd=\boxwidth\textwidth,ht=\boxheight\textheight,sep=1em]{wirkung}
            \scriptsize
            \begin{itemize}
                \item wirkt hemmend in der Area postrema (einem Teil des Brechzentrums)
                \item hat eine starke Affinität zum D2-Rezeptor
                \item \textbf{=Antiemetikum}
                \item gute Wirkung bei PONV (postoperative nauese and vomiting)
            \end{itemize}
        \end{beamercolorbox} & 
        \begin{beamercolorbox}[wd=\boxwidth\textwidth,ht=\boxheight\textheight,sep=1em]{nebenwirkung}
            \begin{itemize}
                \item v.a. bei Hypovolämie ist mit einem Blutdruckabfall zu rechnen
                \item potenziert die Wirkung aller zentral wirkender Pharmaka
            \end{itemize}
        \end{beamercolorbox} \\
    \end{tabular}
\end{frame}

\begin{frame}{
    \textbf{Dehydrobenzperidol}
    \textit{(PONVeridol)}
}
    \begin{tabular}{c c}
        \begin{beamercolorbox}[wd=\boxwidth\textwidth,ht=\boxheight\textheight,sep=1em]{dosierung}
            \begin{itemize}
                \item 0,625-1,25mg
            \end{itemize}
        \end{beamercolorbox} & 
        \begin{beamercolorbox}[wd=\boxwidth\textwidth,ht=\boxheight\textheight,sep=1em]{zusammensetzung}
            \begin{itemize}
                \item 1ml enthält 1,25mg Droperidol
            \end{itemize}
        \end{beamercolorbox} \\
        \multicolumn{2}{}{}
        \begin{beamercolorbox}[wd=\textwidth,ht=\boxheight\textheight,sep=1em]{bemerkung}
            Wechselwirkungen
            \begin{itemize}
                \item Vorsich bei Morbus Parkinson, Bradykardie, Hypokaliämie und Depression
            \end{itemize}
        \end{beamercolorbox} \\
    \end{tabular}
\end{frame}