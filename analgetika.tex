
\begin{frame}{
    \textbf{Acetylsalicylsäure}
    \textit{(Aspirin, Aspisol)}
}
    \begin{tabular}{c c}
        \begin{beamercolorbox}[wd=\boxwidth\textwidth,ht=\boxheight\textheight,sep=1em]{indikation}
            \begin{itemize}
                \item Thrombozytenaggregations-hemmung
                \item Schmerzzustände, auch bei koronarer Herzkrankheit
            \end{itemize}
        \end{beamercolorbox} & 
        \begin{beamercolorbox}[wd=\boxwidth\textwidth,ht=\boxheight\textheight,sep=1em]{kontraindikation}
            \scriptsize
            \begin{itemize}
                \item Acetylsalicylsäure Unverträglichkeit/Allergie
                \item Magen-/Darmgeschwüre
                \item Krankhaft erhöhte Blutungsneigung
                \item Leber-/Nierenversagen
                \item Schwangerschaft (relative KI)
            \end{itemize}
        \end{beamercolorbox} \\
        \begin{beamercolorbox}[wd=\boxwidth\textwidth,ht=\boxheight\textheight,sep=1em]{wirkung}
            \scriptsize
            \begin{itemize}
                \item Gerinnungshemmung (Thrombozytenaggregationshemmer)
                \item analgetisch (schmerzstillen)
                \item antipyretisch (fiebersenkend)
                \item antiphlogistisch (entzündungshemmend)
            \end{itemize}
        \end{beamercolorbox} & 
        \begin{beamercolorbox}[wd=\boxwidth\textwidth,ht=\boxheight\textheight,sep=1em]{nebenwirkung}
            \scriptsize
            \begin{itemize}
                \item Magenbeschwerden
                \item Blutungen
                \item Bei Überempfindlichkeit Bronchokonstriktion
                \item Reye-Syndrom bei Kindern (extrem selten)
                \item Asthmatiker: 8-20\% Intoleranz
            \end{itemize}
        \end{beamercolorbox} \\
    \end{tabular}
\end{frame}

\begin{frame}{
    \textbf{Acetylsalicylsäure}
    \textit{(Aspirin, Aspisol)}
}
    \begin{tabular}{c c}
        \begin{beamercolorbox}[wd=\boxwidth\textwidth,ht=\boxheight\textheight,sep=1em]{dosierung}
            \begin{itemize}
                \item 100-500mg Tabletten, Ampullen, Brausetabletten
            \end{itemize}
        \end{beamercolorbox} & 
        \begin{beamercolorbox}[wd=\boxwidth\textwidth,ht=\boxheight\textheight,sep=1em]{zusammensetzung}
            \begin{itemize}
                \item 1 Inj. Flasche mit 0,5g Acetylsalicysäure (Trockensubstanz)
                \item 1 Ampulle mit 5ml Wasser für Injektionszwecke (Lösungsmittel)
            \end{itemize}
        \end{beamercolorbox} \\
        \multicolumn{2}{}{}
        \begin{beamercolorbox}[wd=\textwidth,ht=\boxheight\textheight,sep=1em]{bemerkung}
        \end{beamercolorbox} \\
    \end{tabular}
\end{frame}

\begin{frame}{
    \textbf{Diclofenac}
    \textit{(Voltaren, Voltadol)}
}
    \begin{tabular}{c c}
        \begin{beamercolorbox}[wd=\boxwidth\textwidth,ht=\boxheight\textheight,sep=1em]{indikation}
            \begin{itemize}
                \item Bei Schmerzen von Gelenken und Entzündungen, z.B.: Gicht, Rheuma
            \end{itemize}
        \end{beamercolorbox} & 
        \begin{beamercolorbox}[wd=\boxwidth\textwidth,ht=\boxheight\textheight,sep=1em]{kontraindikation}
            \tiny
            \begin{itemize}
                \item Dichlorphenylaminophenylacetat Unverträglichkeit/Allergie
                \item Magenulcera
                \item Asthma
                \item Letztes Schwangerschaftsdrittel
                \item Erhöhtes Blutungsrisiko
                \item Niereninsuffizienz
                \item Nicht empfohlen unter dem 15. Lebensjahr
                \item Kardiovaskuläre Vorerkrankungen
            \end{itemize}
        \end{beamercolorbox} \\
        \begin{beamercolorbox}[wd=\boxwidth\textwidth,ht=\boxheight\textheight,sep=1em]{wirkung}
            \scriptsize
            \begin{itemize}
                \item Hemmt unspezifisch die Cyclooxygenase COX 1 und 2
                \item Periphere und zentrale Wirkung
                \item Analgetische, antipyretische und antiphlogistische Wirkung
                \item Thrombozytenaggregationshemmer
            \end{itemize}
        \end{beamercolorbox} & 
        \begin{beamercolorbox}[wd=\boxwidth\textwidth,ht=\boxheight\textheight,sep=1em]{nebenwirkung}
            \scriptsize
            \begin{itemize}
                \item Allergische Reaktion
                \item Asthma, Bronchospasmus
                \item Magenulcera und GI Blutungen
                \item Erhöhte Blutungsneigung
                \item Nierenfunktionsstörung
                \item Kardiale Nebenwirkungen
                \item Hypertonie
                \item Herzinfarkte nach langzeit Einnahme
            \end{itemize}
        \end{beamercolorbox} \\
    \end{tabular}
\end{frame}

\begin{frame}{
    \textbf{Diclofenac}
    \textit{(Voltaren, Voltadol)}
}
    \begin{tabular}{c c}
        \begin{beamercolorbox}[wd=\boxwidth\textwidth,ht=\boxheight\textheight,sep=1em]{dosierung}
            \begin{itemize}
                \item 50-100mg; Maximaldosis: 150-200mg/Tag
                \item i.v., p.o., Kurzinfusion, Zäpfchen, Gel	
            \end{itemize}
        \end{beamercolorbox} & 
        \begin{beamercolorbox}[wd=\boxwidth\textwidth,ht=\boxheight\textheight,sep=1em]{zusammensetzung}
            \begin{itemize}
                \item 1 Ampule 3ml mit 75mg Dichlorphenylaminophenylacetat (15\mgml)
            \end{itemize}
        \end{beamercolorbox} \\
        \multicolumn{2}{}{}
        \begin{beamercolorbox}[wd=\textwidth,ht=\boxheight\textheight,sep=1em]{bemerkung}
        \end{beamercolorbox} \\
    \end{tabular}
\end{frame}

\begin{frame}{
    \textbf{Paracetamol}
    \textit{(Mexalen, Perfalgan, Ben-u-ron)}
}
    \begin{tabular}{c c}
        \begin{beamercolorbox}[wd=\boxwidth\textwidth,ht=\boxheight\textheight,sep=1em]{indikation}
            \begin{itemize}
                \item Kurzzeitbehandlung leich- und mittelstarker Schmerzen
                \item Fieber	
            \end{itemize}
        \end{beamercolorbox} & 
        \begin{beamercolorbox}[wd=\boxwidth\textwidth,ht=\boxheight\textheight,sep=1em]{kontraindikation}
            \begin{itemize}
                \item Paracetamol Unverträglichkeit/Allergie
                \item Schwere Nieren-/ Leberfunktionsstörungen
            \end{itemize}
        \end{beamercolorbox} \\
        \begin{beamercolorbox}[wd=\boxwidth\textwidth,ht=\boxheight\textheight,sep=1em]{wirkung}
            \scriptsize
            \begin{itemize}
                \item Zentrale Hemmung der Prostaglandinsynthese
                \item Analgetische und antipyretische Wirkung, keine antiphlogistische Wirkung
                \item Wirkeintritt nach i.v. Gabe nach ca. 5-10min, Wirkdauer 4-6h
            \end{itemize}
        \end{beamercolorbox} & 
        \begin{beamercolorbox}[wd=\boxwidth\textwidth,ht=\boxheight\textheight,sep=1em]{nebenwirkung}
            \begin{itemize}
                \item Allergische Hautreaktion
                \item Reversible Niereninsuffizienz (sehr selten)
                \item In hoher Dosierung äußerst hepatotoxisch
            \end{itemize}
        \end{beamercolorbox} \\
    \end{tabular}
\end{frame}

\begin{frame}{
    \textbf{Paracetamol}
    \textit{(Mexalen, Perfalgan, Ben-u-ron)}
}
    \begin{tabular}{c c}
        \begin{beamercolorbox}[wd=\boxwidth\textwidth,ht=\boxheight\textheight,sep=1em]{dosierung}
            \begin{itemize}
                \item 500-1000mg p.o. oder i.v., Maximaldosis: 4g/Tag
                \item 15mg/kg KG
                \item Tabletten, i.v., Zäpfchen
            \end{itemize}
        \end{beamercolorbox} & 
        \begin{beamercolorbox}[wd=\boxwidth\textwidth,ht=\boxheight\textheight,sep=1em]{zusammensetzung}
            \begin{itemize}
                \item Perfalgan (10\mgml)
                \item 50ml = 500mg Paracetamol
                \item 100ml = 1000mg Paracetamol
            \end{itemize}
        \end{beamercolorbox} \\
        \multicolumn{2}{}{}
        \begin{beamercolorbox}[wd=\textwidth,ht=\boxheight\textheight,sep=1em]{bemerkung}
            \begin{itemize}
                \item Gute Verträglichkeit
                \item Kein Effekt auf Atmung oder Gerinnung
                \item Kann auch in Schwangerschaft oder Stillzeit angewendet werden (wenn’s sein muss!)
                \item Tageshöchstdosis beachten → Hepatoxizität!
            \end{itemize}
        \end{beamercolorbox} \\
    \end{tabular}
\end{frame}

\begin{frame}{
    \textbf{Metamizol}
    \textit{(Novalgin)}
}
    \begin{tabular}{c c}
        \begin{beamercolorbox}[wd=\boxwidth\textwidth,ht=\boxheight\textheight,sep=1em]{indikation}
            \begin{itemize}
                \item Mittelgradige Schmerzzustände
                \item v.a. viszerale Schmerzen, z.B.: Nieren- und Gallenkoliken	
                \item Therapieresistentes Fieber
            \end{itemize}
        \end{beamercolorbox} & 
        \begin{beamercolorbox}[wd=\boxwidth\textwidth,ht=\boxheight\textheight,sep=1em]{kontraindikation}
            \scriptsize
            \begin{itemize}
                \item Metamizol Unverträglichkeit/Allergie
                \item Schwangerschaft oder Stillzeit
                \item Akute hepatische Porphyrie
                \item Glucose-6-Dehydrogenasemangel (Favabohnenkrankheit)
            \end{itemize}
        \end{beamercolorbox} \\
        \begin{beamercolorbox}[wd=\boxwidth\textwidth,ht=\boxheight\textheight,sep=1em]{wirkung}
            \scriptsize
            \begin{itemize}
                \item Genauer Wirkmechanismus nicht bekannt
                \item Vorwiegend zentrale Wirkung
                \item Analgetisch, antipyretisch, spasmolytisch
                \item Leicht- und mittelgradige aktute und chronische Schmerzen
                \item Wirkeintritt nach ca. 30min
            \end{itemize}
        \end{beamercolorbox} & 
        \begin{beamercolorbox}[wd=\boxwidth\textwidth,ht=\boxheight\textheight,sep=1em]{nebenwirkung}
            \tiny
            \begin{itemize}
                \item Kaum kardiale, renale und gastrointestinale Nebenwirkungen
                \item Allergische Reaktionen bis hin zum Schock (sehr selten)
                \item Agranulozytose (sehr selten)
                    \begin{itemize}
                        \tiny
                        \item Unterschiedliche Angaben bzgl. Inzidenz (1-5/1.000.000); Schweden (1/1400)
                        \item Beginn mit Neutropenie
                        \item Regelmäßige BB-Kontrollen
                    \end{itemize}
                \item Hypotension bei rascher i.v. Gabe
            \end{itemize}
        \end{beamercolorbox} \\
    \end{tabular}
\end{frame}

\begin{frame}{
    \textbf{Metamizol}
    \textit{(Novalgin)}
}
    \begin{tabular}{c c}
        \begin{beamercolorbox}[wd=\boxwidth\textwidth,ht=\boxheight\textheight,sep=1em]{dosierung}
            \begin{itemize}
                \item 500-1000mg p.o. als Tablette 
                \item 3-4x täglich 20 Tropen (=500mg)
                \item 10-20\mgkg KG; Maximaldosis: 4g/Tag
            \end{itemize}
        \end{beamercolorbox} & 
        \begin{beamercolorbox}[wd=\boxwidth\textwidth,ht=\boxheight\textheight,sep=1em]{zusammensetzung}
            \begin{itemize}
                \item 1 Ampulle mit 2ml enthält 1g Metamizol-Natrium (500\mgml)
            \end{itemize}
        \end{beamercolorbox} \\
        \multicolumn{2}{}{}
        \begin{beamercolorbox}[wd=\textwidth,ht=\boxheight\textheight,sep=1em]{bemerkung}
            \begin{itemize}
                \item Verabreichung über 15-30min! (Hypotension)
            \end{itemize}
        \end{beamercolorbox} \\
    \end{tabular}
\end{frame}

\begin{frame}{
    \textbf{Butylscopolamin}
    \textit{(Buscopan)}
}
    \begin{tabular}{c c}
        \begin{beamercolorbox}[wd=\boxwidth\textwidth,ht=\boxheight\textheight,sep=1em]{indikation}
            \begin{itemize}
                \item Krämpfe bei Erkrankungen des Gallenganges und des Darms
                \item Koliken	
            \end{itemize}
        \end{beamercolorbox} & 
        \begin{beamercolorbox}[wd=\boxwidth\textwidth,ht=\boxheight\textheight,sep=1em]{kontraindikation}
            \scriptsize
            \begin{itemize}
                \item Butylscopolamin Unverträglichkeit/Allergie
                \item Tachyarrhythmien
                \item Engwinkelglaukom
                \item Prostataadenom mit Restharnbildung
                \item Stenose des Magen-Darm-Traktes
            \end{itemize}
        \end{beamercolorbox} \\
        \begin{beamercolorbox}[wd=\boxwidth\textwidth,ht=\boxheight\textheight,sep=1em]{wirkung}
            \scriptsize
            \begin{itemize}
                \item Gehört zur Gruppe der Parasympatholytika
                \item Hebt die Wirkung des Parasympathikus auf
                \item Blockiert den muskarinen Acetylcholinrezeptor
                \item Motilitätsminderd auf die glatte Muskulatur
                \item krampflösend
            \end{itemize}
        \end{beamercolorbox} & 
        \begin{beamercolorbox}[wd=\boxwidth\textwidth,ht=\boxheight\textheight,sep=1em]{nebenwirkung}
            \tiny
            \begin{itemize}
                \item Tachykardie (Monitoring!)
                \item Verkürzung der AV Überleitung
                \item Augeninnendruckerhöhung bei Engwinkelglaukom
                \item Mydriasis, Akkomodationsstörungen
                \item Mundtrockenheit
                \item Hemmung der Schweißsekretion, Wärmestau
                \item Miktionsbeschwerden
            \end{itemize}
        \end{beamercolorbox} \\
    \end{tabular}
\end{frame}

\begin{frame}{
    \textbf{Butylscopolamin}
    \textit{(Buscopan)}
}
    \begin{tabular}{c c}
        \begin{beamercolorbox}[wd=\boxwidth\textwidth,ht=\boxheight\textheight,sep=1em]{dosierung}
            \begin{itemize}
                \item 1ml (20mg Butylscopolamin) langsam i.v. oder s.c
                \item Kinder ¼ Ampulle (0,5ml – 10mg Butylscopolamin)	
            \end{itemize}
        \end{beamercolorbox} & 
        \begin{beamercolorbox}[wd=\boxwidth\textwidth,ht=\boxheight\textheight,sep=1em]{zusammensetzung}
            \begin{itemize}
                \item 1 Amp. Zu 1ml enthält 20mg Butylscopolaminbromid (10\mgml)
            \end{itemize}
        \end{beamercolorbox} \\
        \multicolumn{2}{}{}
        \begin{beamercolorbox}[wd=\textwidth,ht=\boxheight\textheight,sep=1em]{bemerkung}
        \end{beamercolorbox} \\
    \end{tabular}
\end{frame}

\begin{frame}{
    \textbf{Ketamin}
    \textit{(Ketanest S)}
}
    \begin{tabular}{c c}
        \begin{beamercolorbox}[wd=\boxwidth\textwidth,ht=\boxheight\textheight,sep=1em]{indikation}
            \begin{itemize}
                \item Kurznarkose
                \item Analgesie
                \item Notfallmedizin
                \item Status asthmaticus	
            \end{itemize}
        \end{beamercolorbox} & 
        \begin{beamercolorbox}[wd=\boxwidth\textwidth,ht=\boxheight\textheight,sep=1em]{kontraindikation}
            \begin{itemize}
                \item Ketamin Unverträglichkeit/Allergie
                \item Herzinfarkt
                \item Ausgeprägte koronare Herzkrankheit
                \item Hypertonie
            \end{itemize}
        \end{beamercolorbox} \\
        \begin{beamercolorbox}[wd=\boxwidth\textwidth,ht=\boxheight\textheight,sep=1em]{wirkung}
            \tiny
            \begin{itemize}
                \item Antagonistische Wirkung am Glutamat-NMDA-Rezeptor
                \item Hemmt die NMDA-abhängige Freisetzung von Acetylcholin und Glutamatrezeptoren
                \item Beinflusst das cholinerge System
                \item Schwach agonistische Wirkung an Opioidrezeptoren
                \item Affinität zu GABA-Rezeptoren
                \item Hemmt die Wiederaufnahme von Noradrenalin und Dopamin an der synaptischen Endplatte
                \item Blockiert den neuronalen Kalziumeinstrom
            \end{itemize}
        \end{beamercolorbox} & 
        \begin{beamercolorbox}[wd=\boxwidth\textwidth,ht=\boxheight\textheight,sep=1em]{nebenwirkung}
            \scriptsize
            \begin{itemize}
                \item v.a. psychotrope Nebenwirkungen
                \item Alpträume und Halluzinationen (10-30\%)
                \item RR-Anstieg
                \item Anstieg der Herzfrequenz
                \item Vermehrter Speichelfluss
            \end{itemize}
        \end{beamercolorbox} \\
    \end{tabular}
\end{frame}

\begin{frame}{
    \textbf{Ketamin}
    \textit{(Ketanest S)}
}
    \begin{tabular}{c c}
        \begin{beamercolorbox}[wd=\boxwidth\textwidth,ht=\boxheight\textheight,sep=1em]{dosierung}
            \tiny
            \begin{itemize}
                \item { 
                    Analgesie 
                    \begin{itemize}
                        \tiny
                        \item Erwachsene 0,1-0,25\mgkg i.v.; 0,25-0,5\mgkg i.m.
                        \item Kinder 0,5-1\mgkg i.v.
                    \end{itemize}
                }
                \item {
                    Anästhesie
                    \begin{itemize}
                        \tiny
                        \item Erwachsene 0,5-1\mgkg i.v.; 2-6\mgkg i.m.
                        \item Kinder 1-2\mgkg i.v.
                    \end{itemize}
                }
                \item {
                    Therapierefraktärer Bronchospasmus
                    \begin{itemize}
                        \tiny
                        \item Erwachsene 0,5-1\mgkg i.v.
                    \end{itemize}
                }
            \end{itemize}
        \end{beamercolorbox} & 
        \begin{beamercolorbox}[wd=\boxwidth\textwidth,ht=\boxheight\textheight,sep=1em]{zusammensetzung}
            \begin{itemize}
                \item Ketanest S: 5\mgml oder 25\mgml Esketamin	
            \end{itemize}
        \end{beamercolorbox} \\
        \multicolumn{2}{}{}
        \begin{beamercolorbox}[wd=\textwidth,ht=\boxheight\textheight,sep=1em]{bemerkung}
            \begin{itemize}
                \item \textbf{CAVE:} Unterschiedliche Konzentrationen im Handel! 5 vs 25 \mgml
            \end{itemize}
        \end{beamercolorbox} \\
    \end{tabular}
\end{frame}

\begin{frame}{
    \textbf{Tramadol}
    \textit{(Tramal)}
}
    \begin{tabular}{c c}
        \begin{beamercolorbox}[wd=\boxwidth\textwidth,ht=\boxheight\textheight,sep=1em]{indikation}
            \begin{itemize}
                \item mittelstarke bis starke Schmerzzustände
            \end{itemize}
        \end{beamercolorbox} & 
        \begin{beamercolorbox}[wd=\boxwidth\textwidth,ht=\boxheight\textheight,sep=1em]{kontraindikation}
            \begin{itemize}
                \item // todo Kontraindikation 
            \end{itemize}
        \end{beamercolorbox} \\
        \begin{beamercolorbox}[wd=\boxwidth\textwidth,ht=\boxheight\textheight,sep=1em]{wirkung}
            \tiny
            \begin{itemize}
                \item schwaches Opioid-Analgetikum
                \item analgetische Potenz: 0,1
                \item vorwiegend nicht selektiver Agonist der $\mu$-, $\kappa$-, $\delta$-Rezeptoren
                \item zentral schmerzhemmende und sedierende Wirkung
                \item Hemmung der Noradrenalinaufnahme und Serotoninfreisetzung
                \item rasche und fast vollständige Resorption
                \item Wirkeintritt nach ca. 2-30min
                \item Wirkdauer: HWZ: 5-6 Stunden
                \item Ausscheidung vor allem über die Niere
            \end{itemize}
        \end{beamercolorbox} & 
        \begin{beamercolorbox}[wd=\boxwidth\textwidth,ht=\boxheight\textheight,sep=1em]{nebenwirkung}
            \begin{itemize}
                \item Schwitzen
                \item Sedierung
                \item ausgeprägt: \textbf{Übelkeit und Erbrechen}
            \end{itemize}
        \end{beamercolorbox} \\
    \end{tabular}
\end{frame}

\begin{frame}{
    \textbf{Tramadol}
    \textit{(Tramal)}
}
    \begin{tabular}{c c}
        \begin{beamercolorbox}[wd=\boxwidth\textwidth,ht=\boxheight\textheight,sep=1em]{dosierung}
            \begin{itemize}
                \item 1-1,5\mgkg KG langsam i.v., ggf. wiederholen
            \end{itemize}
        \end{beamercolorbox} & 
        \begin{beamercolorbox}[wd=\boxwidth\textwidth,ht=\boxheight\textheight,sep=1em]{zusammensetzung}
            \begin{itemize}
                \item 1 Amp. zu 1ml enthält 50mg (50\mgml)
                \item 1 Amp. zu 2ml enthält 100mg (50\mgml) Tramadolhoydrochlorid
            \end{itemize}
        \end{beamercolorbox} \\
        \multicolumn{2}{}{}
        \begin{beamercolorbox}[wd=\textwidth,ht=\boxheight\textheight,sep=1em]{bemerkung}
            \begin{itemize}
                \item // todo Bemerkung
            \end{itemize}
        \end{beamercolorbox} \\
    \end{tabular}
\end{frame}

\begin{frame}{
    \textbf{Nalbuphin}
    \textit{(Nubain)}
}
    \begin{tabular}{c c}
        \begin{beamercolorbox}[wd=\boxwidth\textwidth,ht=\boxheight\textheight,sep=1em]{indikation}
            \begin{itemize}
                \item mittlere Schmerzzustände
                \item vor allem bei Frühgeborenen, Neugeborenen und Kindern
            \end{itemize}
        \end{beamercolorbox} & 
        \begin{beamercolorbox}[wd=\boxwidth\textwidth,ht=\boxheight\textheight,sep=1em]{kontraindikation}
            \begin{itemize}
                \item Atemstörungen
                \item erhöhter intrakranieller Druck
                \item Schwangerschaft
                \item Sulfitallergie
            \end{itemize}
        \end{beamercolorbox} \\
        \begin{beamercolorbox}[wd=\boxwidth\textwidth,ht=\boxheight\textheight,sep=1em]{wirkung}
            \scriptsize
            \begin{itemize}
                \item Opioid-Analgetikum mit zentral schmerzhemmender und sedierender Wirkung
                \item $\kappa$-Agonist/partieller $\mu$-Antagonist
                \item analgetische Potenz: 0,5 - 0,7
                \item Wirkeintritt nach ca. 2-3min
                \item HWZ: 3 Stunden
            \end{itemize}
        \end{beamercolorbox} & 
        \begin{beamercolorbox}[wd=\boxwidth\textwidth,ht=\boxheight\textheight,sep=1em]{nebenwirkung}
            \scriptsize
            \begin{itemize}
                \item Atemdepression (gering, da partieller Antagonist)
                \item Übelkeit, Erbrechen
                \item Miosis
                \item Harnverhalt
                \item Blutdruckabfall
            \end{itemize}
        \end{beamercolorbox} \\
    \end{tabular}
\end{frame}

\begin{frame}{
    \textbf{Nalbuphin}
    \textit{(Nubain)}
}
    \begin{tabular}{c c}
        \begin{beamercolorbox}[wd=\boxwidth\textwidth,ht=\boxheight\textheight,sep=1em]{dosierung}
            \begin{itemize}
                \item 0,1-0,2\mgkgkg; Maximaldosis: 0,25\mgkg
                \item \textbf{Ceiling Effekt}
            \end{itemize}
        \end{beamercolorbox} & 
        \begin{beamercolorbox}[wd=\boxwidth\textwidth,ht=\boxheight\textheight,sep=1em]{zusammensetzung}
            \begin{itemize}
                \item 1 Amp. zu 1ml enthält 10mg Nalbuphin ($10\frac{mg}{ml}$)
                \item 1 Amp. zu 2ml enthält 20mg Nalbuphin ($10\frac{mg}{ml}$)
            \end{itemize}
        \end{beamercolorbox} \\
        \multicolumn{2}{}{}
        \begin{beamercolorbox}[wd=\textwidth,ht=\boxheight\textheight,sep=1em]{bemerkung}
            \begin{itemize}
                \item $\kappa$-Agonist \& partieller $\mu$-Antagonist --> Auslösen von Entzugssysmptomen möglich
            \end{itemize}
        \end{beamercolorbox} \\
    \end{tabular}
\end{frame}

\begin{frame}{
    \textbf{Piritramid}
    \textit{(Dipidolor)}
}
    \begin{tabular}{c c}
        \begin{beamercolorbox}[wd=\boxwidth\textwidth,ht=\boxheight\textheight,sep=1em]{indikation}
            \begin{itemize}
                \item mittelstarke bis starke Schmerzzustände
            \end{itemize}
        \end{beamercolorbox} & 
        \begin{beamercolorbox}[wd=\boxwidth\textwidth,ht=\boxheight\textheight,sep=1em]{kontraindikation}
        // todo Kontraindikation 
        \end{beamercolorbox} \\
        \begin{beamercolorbox}[wd=\boxwidth\textwidth,ht=\boxheight\textheight,sep=1em]{wirkung}
            \begin{itemize}
                \item $\mu$-Rezeptor-Agonist
                \item analgetische Potenz: 0,7
                \item Wirkeintritt nach ca. 10-20min, Wirkdauer: 5-8h, HWZ:4-10h
            \end{itemize}
        \end{beamercolorbox} & 
        \begin{beamercolorbox}[wd=\boxwidth\textwidth,ht=\boxheight\textheight,sep=1em]{nebenwirkung}
            \tiny
            \begin{itemize}
                \item \textbf{Atemdepression}, Sedierung --> Überwachung der Vitalfunktionen
                \item Kreislaufdepression (Hypotonie, Bradykardie)
                \item \textbf{Übelkeit, Erbrechen}
                \item Stimmungsaufhellung
                \item Miosis
                \item Juckreiz
                \item Obstipation, Spasmen der Gallenwege und des Hartraktes
            \end{itemize}
        \end{beamercolorbox} \\
    \end{tabular}
\end{frame}

\begin{frame}{
    \textbf{Piritramid}
    \textit{(Dipidolor)}
}
    \begin{tabular}{c c}
        \begin{beamercolorbox}[wd=\boxwidth\textwidth,ht=\boxheight\textheight,sep=1em]{dosierung}
            \begin{itemize}
                \item 7,5mg-15mg Piritramid s.c., Repetition nach 6 Stunden
                \item 3-4,5mg Piritramid i.v.
                \item bei Kindern 0,1\mgkgkg i.v.
            \end{itemize}
        \end{beamercolorbox} & 
        \begin{beamercolorbox}[wd=\boxwidth\textwidth,ht=\boxheight\textheight,sep=1em]{zusammensetzung}
            \begin{itemize}
                \item 1 Amp. zu 2ml enthält 15mg Piritramid (7,5\mgml)
                \item verdünnen mit 8ml NaCl 0,9\% --> Konzentration von 1,5\mgml
            \end{itemize}
        \end{beamercolorbox} \\
        \multicolumn{2}{}{}
        \begin{beamercolorbox}[wd=\textwidth,ht=\boxheight\textheight,sep=1em]{bemerkung}
        // todo Bemerkung
        \end{beamercolorbox} \\
    \end{tabular}
\end{frame}

\begin{frame}{
    \textbf{Morphin}
    \textit{(Vendal)}
}
    \begin{tabular}{c c}
        \begin{beamercolorbox}[wd=\boxwidth\textwidth,ht=\boxheight\textheight,sep=1em]{indikation}
            \begin{itemize}
                \item schwere Schmerzzustände, terminale Schmerzen
                \item Lungenödem
                \item Myokardinfarkt
            \end{itemize}
        \end{beamercolorbox} & 
        \begin{beamercolorbox}[wd=\boxwidth\textwidth,ht=\boxheight\textheight,sep=1em]{kontraindikation}
            \begin{itemize}
                \item kollikartige Schmerzen
                \item akute Pankreatitis
            \end{itemize}
        \end{beamercolorbox} \\
        \begin{beamercolorbox}[wd=\boxwidth\textwidth,ht=\boxheight\textheight,sep=1em]{wirkung}
            \scriptsize
            \begin{itemize}
                \item zentral schmerzhemmende und sedierende Wirkung
                \item Senkung des $O_2$-Verbrauches, daher ideal beim Myokardinfarkt
                \item analgetische Potenz: 1
                \item Wirkeintritt nach ca. 3-5min, Wirkdauer: 2-5 Stunden, HWZ 3 Stunden
            \end{itemize}
        \end{beamercolorbox} & 
        \begin{beamercolorbox}[wd=\boxwidth\textwidth,ht=\boxheight\textheight,sep=1em]{nebenwirkung}
            \tiny
            \begin{itemize}
                \item \textbf{Atemdepression}
                \item Kreislaufdepression
                \item Übelkeit, Erbrechen
                \item Schwindel, Benommenheit
                \item Harnverhalt
                \item Miosis
                \item Spasmen der Gallenwege und des Hartraktes
                \item Obstipation
                \item Abhängigkeit
            \end{itemize}
        \end{beamercolorbox} \\
    \end{tabular}
\end{frame}

\begin{frame}{
    \textbf{Morphin}
    \textit{(Vendal)}
}
    \begin{tabular}{c c}
        \begin{beamercolorbox}[wd=\boxwidth\textwidth,ht=\boxheight\textheight,sep=1em]{dosierung}
            \begin{itemize}
                \item 0,05-0,1\mgkgkg, Tageshöchstdosis: 100mg
            \end{itemize}
        \end{beamercolorbox} & 
        \begin{beamercolorbox}[wd=\boxwidth\textwidth,ht=\boxheight\textheight,sep=1em]{zusammensetzung}
            \scriptsize
            \begin{itemize}
                \item 1 Amp. 1ml enthält 10mg Morpinhydrochlorid (10\mgml)
                \item 1 Amp. 1ml enthält 20mg Morpinhydrochlorid (20\mgml)
                \item aus getrocknetem Milchsaft der grünen Mohnkapsel gewonnen
            \end{itemize}
        \end{beamercolorbox} \\
        \multicolumn{2}{}{}
        \begin{beamercolorbox}[wd=\textwidth,ht=\boxheight\textheight,sep=1em]{bemerkung}
            \begin{itemize}
                \item Hauptalkaloid im Opium
                \item ältestes und am besten erforschtes Opioid
            \end{itemize}
        \end{beamercolorbox} \\
    \end{tabular}
\end{frame}

\begin{frame}{
    \textbf{Fentanyl}
    \textit{(Fentanyl)}
}
    \begin{tabular}{c c}
        \begin{beamercolorbox}[wd=\boxwidth\textwidth,ht=\boxheight\textheight,sep=1em]{indikation}
            \begin{itemize}
                \item starke Schmerzen
                \item Narkoseeinleitung
            \end{itemize}
        \end{beamercolorbox} & 
        \begin{beamercolorbox}[wd=\boxwidth\textwidth,ht=\boxheight\textheight,sep=1em]{kontraindikation}
        // todo Kontraindikation 
        \end{beamercolorbox} \\
        \begin{beamercolorbox}[wd=\boxwidth\textwidth,ht=\boxheight\textheight,sep=1em]{wirkung}
            \scriptsize
            \begin{itemize}
                \item \textbf{= $\mu$-Agonist}
                \item zentral schmerzhemmende und sedierende Wirkung
                \item \textbf{analgetisch Potenz: 100}
                \item kein Ceiling Effekt
                \item Wirkeintritt nach ca. 2-5 min, Wirkdauer: 30-40 min, HWZ: 2-7 Stunden
            \end{itemize}
        \end{beamercolorbox} & 
        \begin{beamercolorbox}[wd=\boxwidth\textwidth,ht=\boxheight\textheight,sep=1em]{nebenwirkung}
            \scriptsize
            \begin{itemize}
                \item \textbf{Atemdrepression} --> Überwachung der Vitalfunktionen
                \item Müdigkeit
                \item Kreislaufdepression/Blutdruckabfall
                \item Übelkeit, Erbrechen
                \item langsam spritzen --> sonst Hustenreiz!
            \end{itemize}
        \end{beamercolorbox} \\
    \end{tabular}
\end{frame}

\begin{frame}{
    \textbf{Fentanyl}
    \textit{(Fentanyl)}
}
    \begin{tabular}{c c}
        \begin{beamercolorbox}[wd=\boxwidth\textwidth,ht=\boxheight\textheight,sep=1em]{dosierung}
            \begin{itemize}
                \item Analgesie: bis 1,5\mukgkg i.v.
                \item Anästhesie: bis 2-5\mukgkg i.v.
            \end{itemize}
        \end{beamercolorbox} & 
        \begin{beamercolorbox}[wd=\boxwidth\textwidth,ht=\boxheight\textheight,sep=1em]{zusammensetzung}
            \begin{itemize}
                \item 1 Amp. zu 2ml enthält 100\mug (50\mugml)
                \item 1 Amp. zu 10ml enthält 500\mug (50\mugml)
                \item synthetisch hergestellt
            \end{itemize}
        \end{beamercolorbox} \\
        \multicolumn{2}{}{}
        \begin{beamercolorbox}[wd=\textwidth,ht=\boxheight\textheight,sep=1em]{bemerkung}
        // todo Bemerkung
        \end{beamercolorbox} \\
    \end{tabular}
\end{frame}